\documentclass[conference]{IEEEtran}
\IEEEoverridecommandlockouts
% The preceding line is only needed to identify funding in the first footnote. If that is unneeded, please comment it out.
\usepackage{cite}
\usepackage{amsmath,amssymb,amsfonts}
\usepackage{algorithmic}
\usepackage{graphicx}
\usepackage{textcomp}
\usepackage{xcolor}
\def\BibTeX{{\rm B\kern-.05em{\sc i\kern-.025em b}\kern-.08em
    T\kern-.1667em\lower.7ex\hbox{E}\kern-.125emX}}
\begin{document}

\title{Operating System Assignment 1\\
{\footnotesize \textsuperscript{*}Research Task: The structure of OSes (latest stable versions) like iOS, Android, Ubuntu, Windows 10/11, MacOS.}
}

\author{\IEEEauthorblockN{Sulav Baral}
\IEEEauthorblockA{\textit{Fourth Semester, Section A} \\
\textit{Prime College}\\}
}

\maketitle

\begin{abstract}
Research Task: The structure of OSes (latest stable versions) like iOS, Android, Ubuntu, Windows 10/11, MacOS.
\end{abstract}

\begin{IEEEkeywords}
component, formatting, style, styling, insert
\end{IEEEkeywords}

\section{Introduction}
Although it's a hard task to tell the exact defination of the operating system, we can regard an OS as a system software that manages and controls the hardware and software resources of a computer system, providing a common services layer for all other software applications to run on top of. It provides the path or the gateway for the users to run multiple applications as it is responsible for allocating system resources like CPU time, memory and input/output devices which acts as a interface for the user to interact with the computer. Operating system is a very critical component of a computer system which we can say that acts like an intermediate between hardware, software and the user. Operating system are divided into various types, and their uses may vary for a certain type of system. In this research task we are mainly focusing on the operating systems that we use in our daily basis like :

\begin{enumerate}
    \item Android
    \item iOS
    \item Ubuntu 
    \item Windows
    \item MacOS
\end{enumerate}

These are some OSes that we see in our daily lives here and there and some we use it in the daily basis in our own devices. Below we are going to discuss about these OSes structures/architectures in details.

\section{Structure of Operating Systems}

\subsection{Android Operating System}
Android a commonly used open source operating system that powers a huge amount of modern devices. The original creator of the android operating system is Andy Rubin who founded Android Inc. in Palo Alto, California in October 2003 which is mates Rich Miner, Nick Sears and Chris White which was then later acquired by Google in 2005 for just 50 million at that time which is now a multi billion dollar industry under Google. The source code of android is available to the Android Open Source Project(AOSP) and the first source code was released back in 2007. Since then the development of the android operating of system has been immaculate. Starting from the version 1.0 now we have now reached at the 13th version of the operating system which has been polished every update to make it a perfect operating system. Many companies use the android as the base system which they add their features upon the make it custom but at the core it's the android which powers these phones. Android makes up about 71.74\% of the total mobile operating system market according to Google. At the core the programming languages used to make up the android ecosystem are Java, C, C++, Kotlin. On recent developments other languages which the introduction of different frameworks now can be used for cross platform development like React Native for JavaScript, Flutter/Dart and many more.

\begin{figure}[htbp]
\centerline{\includegraphics[scale=0.5]{Android_open_source_project.png}}
\caption{Structure of Android Operating System}
\label{fig}
\end{figure}
\subsubsection{Structure}
\begin{itemize}
\item Kernal:\linebreak We all know that kernel is the core or the central component of any kind of Operating System. It is the primary interface between the hardware and the processes of the computer.  At the core the android kernel is based on an upstream Linux Long Term Supported (LTS) kernel. The LTS kernels are combined with Android-specific patches to form Android Common Kernels (ACKs). The type of kernel that’s used in the Android Operating system is the Monolithic Linux Kernel. The main reason to use a Linux Kernel is as Linux is a very portable operating system which can be built and ran in variety of devices. The kernel here used is a modified version of the Linux Kernel, with additional features added to support the mobile devices. Starting from Android 11 the ACKs are used to produce Generic Kernal Images(GKIs). These are 64 bit ARM kernels that can be used in any device with vendor-supplied SoC and driver support which helps the OEMs (Original equipment manufacturers) as they don’t have to deal with the kernel and it requires less maintenance.
\linebreak
\item HAL/HIDL:\linebreak HAL(Hardware Abstraction Layer) and HIDL(HAL Interface Definition Language) which are one of the components of the architecture are two of the important layers that build up the Android Operating System which provide a standardized interface for communication between the hardware and the software layers of the system. Between the hardware drivers specific to each device and the Android OS framework is the HAL layer. By offering an abstraction layer, it enables the Android framework to interact with the hardware drivers without having to be aware of the precise specifications of the underlying hardware. The HAL layer offers a standardized set of APIs that the Android framework uses to communicate with the hardware, such as the camera, sensors, and input devices. As of Android 10, HIDL is deprecated and Android is migrating to use AIDL everywhere. HIDL is intended to be used for inter-process communication (IPC). HALs created with HDL are called binderized HALs in that they can communicate with other architecture layers using binder inter-process communication (IPC) calls. Usage of HAL/HIDL provides various advantages like portability, abstraction, modularity, compatibility and many others.
\linebreak
\item{Android\;Runtime} :\linebreak The android runtime is a virtual machine in android. This layer is responsible for the execution of the compiled code of android applications. When the user runs the application then the bytecode written in .dex files are translated by the Android Runtime into the machine code which can be understood by the machine. Upto Android 4.4 Dalvik was used as the virtual machine which was then replaced by ART(android runtime) as the replacement of the Dalvik virtual machine after the release of Android Lollipop. The main difference between these two types of VM is that Dalvik used JIT(Just in time) compilation where as the ART used AOT(ahead of time) compilation. JIT means that the code is translated when the application is ran where as the AOT compilation compiles the code as the software/application is being installed. The AOT approach hugely improved the runtime performance. However there were some drawbacks like apps taking time to get installed, slower system updates and high memory usage. But in the long run as the devices got more powerful this approach made the overall experience better than its predecessor. Other benefits like profile-guided optimization(introduced in android nougat) which helped the optimization as the app ran in common usage patterns. Also AOT provides improved garbage collection which reduces the amount of memory used and boosts the overall performance of the device. 
\linebreak
\item Native\;Libraries: \linebreak Android applications can contain compiled native libraries. A native library is code that a developer writes and then compiles for a particular computer architecture. Most of the time that means code written in C or C++. They provide additional functionality or access to system resources not available through the Java-based Android application framework. The Java Native Interface (JNI) allows developers to declare Java methods implemented in native code (usually compiled C/C++). The JNI interface is not specific to the Android operating system, but applies more generally to Java applications running on different platforms. Together, JNI and NDK allow Android developers to implement some features in their applications in native code. Native libraries provide several benefits like:

\begin{itemize}
    \item Access to system resources
    \item Performance
    \item Reusability
\end{itemize}

Some popular Native Libraries used in android operating system are OpenCV, webkit, SQLite etc. 
\linebreak
\item Frameworks: \linebreak
The android framework is the set of API's that allow developers to quickly and easily write apps for android phones. It consists of tools for designing UIs like buttons, text fields, image panes, and system tools like intents (for starting other apps/activities or opening files), phone controls, media players, ect. Essentially an android app consists of Activities (programs that the user interacts with), services (programs that run in the background or provide some function to other apps), and broadcast receivers (programs that catch information important to your app). Portions of the framework are publicly accessible through the use of the Android API. Other portions of the framework are available only to OEMs through the use of the system APIs. Android framework code runs inside an app's process. By providing pre-built components and APIs for common tasks, frameworks make it easier for developers to create Android applications and help to ensure that applications are consistent and behave in a predictable manner.

\end{itemize}

\subsection{iOS Operating System}

After android, the second most popular operating system is the iOS operating system used in iPhones. iOS is also the base for other OS used in apple devices like watchOS, tvOS and iPadOS. This operating system was unveiled in 2007 with the first iPhone released by Apple back then and has continuously evolved to become one of the mostly used operating system right now in the world. The iOS operating system is very well optimized and people love the iOS for its ecosystem and tremendous optimization to the software and both at the hardware level. During the making of this report the latest iOS released stably by Apple is 16.3.1. According to the reports, iOS powers over 27.6 percentage of the total mobile devices making it the second mostly used operating system in terms of handheld cell phones. 


\subsection{Ubuntu Operating System}

Ubuntu a free and open source Linux Distribution based on Debian is one of the most popular distributions in the Linux world. Backed up by a tech giant Canonical Ubuntu has made it's name as one of the best distros that people can use to experience Linux in their personal computers. Ubuntu comes in three editions that are Desktop, Server and Core meaning that this distribution can be used in a variety of devices including personal computers, servers and  IoT devices as well. The latest version of the Ubuntu operating system for desktop PCs and laptops, Ubuntu 22.10 comes with nine months of security and maintenance updates, until July 2023.

\begin{figure}[htbp]
\centerline{\includegraphics[scale=0.5]{architecture-of-linux.png}}
\caption{Structure of Linux}
\label{fig2}
\end{figure}

\begin{itemize}
    \item Kernel Layer

    At the core of the Ubuntu operating system is the Linux kernel. The kernel is responsible for managing hardware resources such as memory, CPU, and input/output devices. It also provides security features such as process isolation, memory protection, and inter-process communication. The Linux kernel is the backbone of the Ubuntu operating system, and it plays a crucial role in its stability and security.
    The Ubuntu kernel is a modified version of the Linux kernel that is optimized for the Ubuntu operating system. It includes features such as AppArmor, which provides mandatory access control for applications, and the Upstart init system, which provides a faster boot time and improved system responsiveness. The Ubuntu kernel is regularly updated with security patches and bug fixes to ensure the stability and security of the operating system.
    \linebreak
    \item System Libraries Layer

    The System Libraries layer contains various libraries that provide a set of functions and interfaces to the applications. These libraries include GNU C Library (glibc), which provides basic system calls and C language support, and OpenSSL, which provides support for secure communication over the internet. Other libraries include GTK+, Qt, and SDL, which provide support for graphical user interfaces and multimedia applications.

    The System Libraries layer also includes the X Window System, which provides a graphical interface for the Ubuntu operating system. The X Window System consists of various components such as the X server, which manages the display hardware, and the window manager, which manages the windows and user interface elements. The X Window System is highly configurable, and it allows users to customize the appearance and behavior of the Ubuntu operating system.
    \linebreak
    \item User Space Layer

    The User Space layer is where the applications and user interfaces reside. It includes various system utilities and applications such as the GNOME desktop environment, which provides a user-friendly graphical interface for the Ubuntu operating system. It also includes various command-line utilities and text editors such as Vim and Nano, which provide a powerful interface for developers and system administrators.

    The User Space layer also includes a number of software applications such as LibreOffice, Firefox, and Thunderbird, which provide productivity tools for users. These applications are installed using the Debian package management system, which is a powerful tool for managing software packages on Ubuntu operating system.
    \linebreak
    \item  Package Management System

    Ubuntu uses the Debian package management system for managing software packages. The package management system is responsible for installing, updating, and removing software packages. It also resolves dependencies between packages and manages the configuration files associated with the packages. The package management system ensures that the Ubuntu operating system is up-to-date and secure by providing regular security updates and bug fixes.

    The Debian package management system is highly scalable and flexible, and it allows users to easily manage software packages on their Ubuntu operating system. The package management system includes various tools such as dpkg, which is used for installing and removing packages, and apt-get, which is used for managing package dependencies and updating the system.

\end{itemize}
\subsubsection*{Conclusion}
In conclusion, the architecture of Ubuntu operating system is based on the Linux kernel, and it is designed to provide a stable, secure, and user-friendly platform for computing. The kernel layer provides a solid foundation for managing hardware resources and security features. The System Libraries layer provides a set of functions and interfaces to the applications, while the User Space layer provides a user-friendly interface for the end-users. The package management system ensures that the Ubuntu operating system is up-to-date and secure by providing regular security updates and bug fixes.

\subsection{Windows Operating System}
Windows developed by Microsoft is the most popular operating system of all time which takes about 75\% as of reports in April 2022 of the total market share of the operating systems used in the devices. It is known for its user-friendly interface, widespread availability, and compatibility with a vast range of hardware and software. The latest stable versions of Windows are Windows 10 and Windows 11. The simplified structure of the windows architecture is shown in the figure below which consists of mainly two parts the user mode and the kernal mode.

\begin{figure}[htbp]
\centerline{\includegraphics[scale=0.5]{winbows.png}}
\caption{Structure of Windows OS}
\label{fig3}
\end{figure}

\begin{itemize}
    \item Kernel Mode: In kernel mode, the operating system has full access to system resources and can execute any instruction, including privileged instructions that are restricted in user mode. The kernel is responsible for managing the system's resources and executing system services. It is divided into several components:
     \begin{itemize}
    
         \item Kernel: At the core of the Windows operating system is the Kernel layer. The kernel is responsible for managing essential hardware resources such as memory, CPU, and input/output devices. It also provides security features such as process isolation, memory protection, and inter-process communication. The Windows kernel is the backbone of the Windows operating system, and it plays a crucial role in its stability and security. The Windows kernel is known as the Windows NT kernel in general

        The Windows kernel is a hybrid kernel, which means it includes features of both monolithic and micro kernels. It includes various components such as the Executive, which manages system resources, and the Memory Manager, which manages memory resources.
        \linebreak
        \item Hardware Abstraction Layer:

        The Hardware Abstraction Layer (HAL) is the layer between the kernel and the hardware of the computer. It provides a standardized interface for the kernel to access hardware resources such as the CPU, memory, and input/output devices. The HAL ensures that the Windows operating system can run on different hardware platforms without requiring modifications to the kernel.
        \linebreak
        \item Device Drivers:
        Device drivers are software components that allow the operating system to communicate with hardware devices such as printers, scanners, and network adapters. These drivers provide a consistent interface for the operating system to access the hardware, regardless of the specific device. Device drivers are loaded into the kernel and run in kernel mode to provide direct access to hardware devices.
     \end{itemize}
    
     \item User Mode: The User Mode layer of the Windows operating system consists of various subsystems and services that run in user mode. These subsystems and services provide support for various features such as networking, printing, and multimedia. The User Mode layer includes various components such as the Win32 subsystem, which provides support for running Win32 applications, and the Graphics Device Interface (GDI), which provides support for graphics rendering.

     \begin{itemize}
         \item Environment Subsystem: The environment subsystem is a component of the user mode that provides a runtime environment for running applications. It includes the subsystems for running Win32 applications, POSIX applications, and OS/2 applications. Each subsystem provides a set of APIs and services that enable applications to interact with the operating system and hardware resources. The environment subsystem also includes the Console subsystem, which provides a command-line interface for users to interact with the system.2d
         \linebreak
         \item System Processes/Services: System processes and services are programs that run in the background and provide system-level functionality to user-mode applications. They are managed by the Service Control Manager (SCM) and can be started, stopped, and configured through the Services console. Examples of system processes and services include the Print Spooler, the Task Scheduler, and the Windows Update service.
         \linebreak
         \item Applications: Applications are programs that run in user mode and provide a wide range of functionality to users. They can be written in various programming languages such as C++, C\#, and Java, and can use APIs provided by the operating system to interact with system resources. Examples of applications include web browsers, media players, and office productivity software.
     \end{itemize}
\end{itemize}

\subsubsection*{Conclusion}
In conclusion, the structure of the Windows operating system is designed to provide a robust and flexible platform for building applications for personal computers and other devices. The Kernel layer provides a stable foundation, while the HAL, User Mode, Shell, and Application layers provide support for various hardware and software features. By dividing the system into distinct layers, each with its own role and set of responsibilities, Windows is able to provide a stable and flexible platform for developers and users alike.

\subsection{Mac Operating System}

\begin{figure}[htbp]
\centerline{\includegraphics[scale=0.2]{1024px-Diagram_of_Mac_OS_X_architecture.svg.png}}
\caption{Structure of Mac OS}
\label{fig4}
\end{figure}

\end{document}
